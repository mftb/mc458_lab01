\documentclass[11pt]{article}
\usepackage[brazil]{babel}
\usepackage[utf8]{inputenc}
\usepackage{epsfig}
\usepackage{setspace}
\usepackage{setspace}
\usepackage{float}

\oddsidemargin 0pt  % was 38
\evensidemargin 0pt % was 38
\marginparwidth 0pt % was 68

\topmargin 30pt   % was 27
\headheight 0pt  % was 12
\headsep 0pt     % was 25
%\footheight 0pt  % was 12
\footskip 30pt   % was 30

\textwidth 470pt      % was 390pt
\textheight 620.5pt  % was 536.5

\floatstyle{ruled}
\newfloat{algoritmo}{thp}{loa}
\floatname{algoritmo}{Algoritmo}
 
\input{abaco.tex}

\begin{document}

\thispagestyle{empty}

\begin{minipage}[tl]{31mm}
  \ABACO{1}{9}{6}{9}{1}
\end{minipage}
\hspace*{3mm}
\begin{minipage}[tl]{12cm}
  \begin{center}
    {
      {\Large Laboratório de Sistemas de Computação \\
        Departamento de Sistemas de Computação \\ 
        Instituto de Computação \\
        Universidade Estadual de Campinas}
    }
  \end{center}
\end{minipage}

{\LARGE
  \begin{center}
      \textbf{Súmula Curricular}

      \textbf{Rodolfo Jardim de Azevedo - MS-5}

  \end{center}
}

\section{Produção Científica}

\subsection{Publicações (10 mais importantes relacionadas ao tema)}

\begin{enumerate}


\item Garcia, Maxiwell; Rigo, Sandro; Azevedo, Rodolfo. Optimizing a
  Retargetable Compiled Simulator to Achieve Near-Native
  Performance. In: Simpósio de Sistemas Computacionais - WSCAD-SSC,
  2010, Petrópolis. WSCAD-SSC 2010 - Simpósio de Sistemas
  Computacionais, 2010.

\item Leonardo Ecco; Bruno Lopes; Pannain, Ricardo; Centoducatte,
  Paulo Cesar; Azevedo, Rodolfo . SPARC16: A New Compression Approach
  for the SPARC Architecture. In: International Symposium on Computer
  Architecture and High-Performance Computing - SBAC-PAD, 2009, São
  Paulo. Proceedings of International Symposium on Computer
  Architecture and High Performance Computing - SBAC-PAD,
  2009. p. 169-176.

\item Klein, Felipe Vieira; Leão, R.; Araújo, Guido Costa Souza de;
  Santos, Luiz Cláudio Villar dos; Azevedo, Rodolfo. A Multi-Model
  Engine for High-Level Power Estimation Accuracy Optimization. IEEE
  Transactions on Very Large Scale Integration Systems, v. 17,
  p. 660-673, 2009.
  
\item Baldassin, A.; Klein, Felipe Vieira; Centoducatte, Paulo Cesar;
  Azevedo, Rodolfo; Araújo, Guido Costa Souza de. Characterizing the
  Energy Consumption of Software Transactional Memory. IEEE Computer
  Architecture Letters, v. 8, p. 1-4, 2009.

\item Santos, Ricardo Ribeiro dos; Batistella, Rafael Fernandes;
  Azevedo, Rodolfo. A Pattern Based Instruction Encoding Technique for
  High Performance Architectures. International Journal of High
  Performance Systems Architecture (Print), v. 2, p. 71-80, 2009.

\item Azevedo, Rodolfo; Rigo, Sandro; Bartholomeu, Marcus; Araújo,
  Guido Costa Souza de; Araújo, Cristiano; Barros, Edna. The ArchC
  Architecture Description Language and Tools. International Journal
  of Parallel Programming, v. 33, n. 5, p. 453-484, 2005.

\item Rigo, Sandro; Araújo, Guido Costa Souza de; Bartholomeu, Marcus;
  Azevedo, R. J.. ArchC: A SystemC-Based Architecture Description
  Language. In: 16th Symposium on Computer Architecture and High
  Performance Computing, 2004, Foz do Iguaçu. SBAC-PAD'04,
  2004. p. 66-73.

\item Bartholomeu, M.; Azevedo, R. J.; Rigo, Sandro; Araújo, Guido
  Costa Souza de. Optimizations for compiled simulation using
  instruction type information. In: 16th Symposium on Computer
  Architecture and High Performance Computing, 2004, Foz do
  Iguaçu. SBAC-PAD'04, 2004. p. 74-81.

\item Rigo, Sandro; Juliato, M. R.; Azevedo, R. J.; Araújo, Guido
  Costa Souza de; Centoducatte, Paulo Cesar. Teaching Computer
  Architecture Using an Architecture Description Language. In:
  Workshop on Computer Architecture Education, 2004, Munich. Workshop
  on Computer Architecture Education, 2004. p. 22-28.
  
\item Borin, E.; Klein, Felipe Vieira; Moreano, N. B.; Azevedo, R. J.;
  Araújo, Guido Costa Souza de. Fast Instruction Set
  Customization. In: 2nd Workshop on Embedded Systems for Real-Time
  Multimedia, 2004, Estocolmo. ESTIMedia 2004, 2004.
 
\end{enumerate}

\subsection{Resumo}

\begin{table}[htbp]
  \centering
  \begin{tabular}{|l|r|}
\hline
Item & Quantidade \\ \hline \hline
Livros Publicados & 1 \\ \hline
Publicações em Periódicos com Seletiva Política Editorial & 9 \\
\hline
Capítulos de Livros & 2 \\ \hline
Artigos em Conferências & 45 \\ \hline
Dissertações de Mestrado Orientadas e já defendidas & 20 \\ \hline
Teses de Doutorado Orientadas e já defendidas & 3 \\ \hline
Teses de Doutorado Co-Orientadas e já defendidas & 2 \\ \hline

  \end{tabular}
  \label{tab:resumo}
\end{table}

\section{Atividades Atuais}

\subsection{Auxílios à Pesquisa Vigentes}

\begin{itemize}
\item  Projeto Universal do CNPq - 2010 sob coordenação do Prof. Guido Araújo.
\end{itemize}

\section{Outras Informações}

A partir de setembro/2006 recebi uma bolsa de Produtividade em
Pesquisa nível 2 do CNPq.

No momento acabo de retornar de um período sabático na Universidade de
Washington, EUA. Justamente pelo afastamento, no momento não tenho
nenhum projeto com financiamento ativo como coordenador. Estou
trabalhando para submeter projetos nas próximas chamadas do CNPq e
também para a FAPESP, no meio de 2012.


\end{document}

