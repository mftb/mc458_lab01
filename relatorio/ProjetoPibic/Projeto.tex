\documentclass[10pt,a4paper]{article}

\usepackage [utf8]{inputenc}
\usepackage[brazil]{babel}
\usepackage[T1]{fontenc}
%\usepackage{geometry}                % See geometry.pdf to learn the layout options. There are lots.
\usepackage{graphicx}
\usepackage{amssymb}
\usepackage{epstopdf}
\usepackage{indentfirst}
\usepackage{fancyhdr}
\usepackage{subfig}
\usepackage{wrapfig}
\usepackage{makeidx}
\usepackage[usenames]{color}
\newcommand{\todo}[1]{\textcolor{red}{\footnote{\textcolor{red}{TODO: #1}}}}
\usepackage{multirow}
\usepackage{rotating}
\usepackage{lmodern}
\usepackage{indentfirst}
\usepackage{array}

%%% usado para codigo %%%
\usepackage{listings}
\lstset{
%  language = C,
  basicstyle=\footnotesize\ttfamily, 
  numbers=left,               
  numberstyle=\tiny,         
  %stepnumber=2,              
  numbersep=5pt,             
  tabsize=2,                  
  extendedchars=true,  
  breaklines=true,       
  keywordstyle=\color{blue},
%  frame=b,         
  stringstyle=\color{green}\ttfamily, 
  showspaces=false,          
  showtabs=false,             
  xleftmargin=17pt,
  framexleftmargin=17pt,
  framexrightmargin=5pt,
  framexbottommargin=4pt,
  %backgroundcolor=\color{lightgray},
  showstringspaces=false              
}
\lstloadlanguages{C}
\usepackage{caption}
\DeclareCaptionFont{white}{\color{white}}
\DeclareCaptionFont{green}{\color{green}}
\DeclareCaptionFormat{listing}{\colorbox[cmyk]{0.43, 0.35, 0.35,0.01}{\parbox{\textwidth}{\hspace{15pt}#1#2#3}}}
\captionsetup[lstlisting]{format=listing,labelfont=white,textfont=white, singlelinecheck=false, margin=0pt, font={bf,footnotesize}}


\begin{document}
%%%%%%%%%%%%%%%%%%%%%%%%%%%%%%%%%
%%%%%                                Capa                                    %%%%%
%%%%%%%%%%%%%%%%%%%%%%%%%%%%%%%%%
    \begin{titlepage}
        \begin{center} \line(5,0){350} \end{center}
        \begin{center} \huge{ArchC Platform Manager \\ Um gerenciador de pacotes para plataformas de sistemas embarcados} \end{center}
        \begin{center} \line(5,0){350} \end{center}
        \vspace{5cm}
        \begin{center} \large{Instituto de computação - UNICAMP} \end{center}
        \begin{center} Campinas, \today \end{center}
        \vspace{3cm}
        \begin{center} \large{Matheus Ferreira Tavares Boy} \end{center}
        
    \end{titlepage}
    \tableofcontents
    \clearpage
    
        
%%%%%%%%%%%%%%%%%%%%%%%%%%%%%%%%%
%%%%%                                Introduçãoo                            %%%%%
%%%%%%%%%%%%%%%%%%%%%%%%%%%%%%%%%    
\section{Introdução}

Este projeto visa atualizar e incrementar o atual gerenciador de
componentes de plataformas utilizado no projeto ArchC. Atualmente o
gerenciador de plataformas é denominado ARP e pretendemos com a nova
versão, que será chamada de ACPM, incluir as seguintes
funcionalidades:

\begin{enumerate}
\item Suporte a repositórios remotos, de forma a facilitar a busca por
  componentes para novas plataformas. Ao incluir esta funcionalidade,
  integraremos um conjunto de componentes já existentes na criação de
  um repositório global.

\item Melhor empacotamento dos componentes, de forma a facilitar tanto
  a identificação quanto o processo de busca individual. Atualmente a
  ARP só permite buscar uma plataforma inteira, pretende-se que seja
  possível buscar apenas um periférico.

\item Suporte à instalação de ferramentas na estrutura do ACPM,
  incluindo \textit{scripts} de instalação, facilitando o trabalho dos
  usuários do sistema.

\item Criar uma interface gráfica via web para utilização tanto do
  gerenciador de pacotes quanto também para gerar o arquivo principal
  das novas plataformas.

\item Suporte no desenvolvimento do ESLBench, um benchmark de
  plataformas que está sendo desenvolvido no LSC, pela aluna de
  doutorado Liana Duenha. O candidato deste projeto já está
  trabalhando deste o início deste ano em funcionalidades de
  integração do ESLBench.

\end{enumerate}
        
%%%%%%%%%%%%%%%%%%%%%%%%%%%%%%%%%
%%%%%                        Conceitos básicos                        %%%%%
%%%%%%%%%%%%%%%%%%%%%%%%%%%%%%%%%
\section{Conceitos básicos}

Esta seção descreve os conceitos básicos necessários para o
desenvolvimento deste projeto. Uma definição que se faz importante no
primeiro momento é o conceito de plataforma. No contexto deste
projeto, uma plataforma é um conjunto de componentes de hardware que
incluem, no mínimo, um processador e um periférico, preferencialmente
tendo vários periféricos interligados e também algum mecanismo de
interconexão entre eles (barramento, NoC, etc). A existência de um
processador garante que a plataforma seja programável e, portanto,
possa ter uma finalidade mais genérica, onde cada software diferente
pode especializa-la para um uso distinto.

\subsection{\textit{ARP - ArchC Reference Platform}}
%----------------------------------------------------------------------------------------
O \textit{ARP}\cite{arp} é uma ferramenta de gerenciamento de pacotes para
geração de plataformas de sistemas embarcados modelados em
\textit{ArchC}\cite{archc,TR-IC-03-15,Rigo2004SBAC}. Faz parte do projeto \textit{ArchC} e foi desenvolvido
no Laboratório de Sistemas de Computação do Instituto de Computação da
Unicamp.

O objetivo principal do \textit{ARP} é organizar e facilitar a geração
de plataformas. O \textit{ARP} organiza os componentes da plataforma
numa estrutura de diretórios própria de modo que seja possível criar
diversas conformações de plataforma sem gerar redundância entre os
componentes. Desse modo,
cada componente é compilado apenas uma vez e utilizado pela plataforma
como uma biblioteca. Tanto os componentes quanto as plataformas são
descritos utilizando-se \textit{ArchC} ou \textit{SystemC}.

Isso se dá devido a uma separação física na estrutura de
arquivos entre:

\begin{description}
\item{Processadores:} Neste diretório ficam os processadores que podem
  ser utilizados nas plataformas. Mesmo que seja necessário modelar um
  sistema multi-core, apenas uma cópia do processador será compilada e
  a biblioteca será instanciada o número de vezes necessário.

\item{IS:} Neste diretório ficam os componentes que fazerm a
  interligação entre os demais componentes de hardware do sistema
  (\textit{Interconection Structures}). Normalmente são barramentos,
  NoCs ou roteadores. Alguns destes componentes precisam ser
  especializados estaticamente com o mapeamento de endereços da
  plataforma.

\item{\textit{Wrappers}:} Estes são os componentes que fazem a
  conversão de protocolos de comunicação entre os demais módulos da
  plataforma. Uma conversão de protocolo pode incluir a mudança no
  nível de abstração da comunicação (uma versão comportamental se
  comunicando com uma RTL).

\item{IP:} Estes são os periféricos em geral da
  plataforma, os demais módulos de hardware necessários.

\item{Software:} Este diretório terá todos os softwares que podem
  executar nas plataformas. Não é obrigatório que qualquer software
  possa ser executado em qualquer plataforma, apenas que eles fiquem
  num ponto específico da estrutura de diretórios para facilitar a
  compilação.

\item{Platform:} Esta é a pasta que contém o arquivo principal da
  plataforma, indicando os componentes necessários para execução e
  também interligando-os da forma desejada. É neste diretório que é
  montado o executável da plataforma.

\end{description}

%----------------------------------------------------------------------------------------    
\subsection{\textit{ACPM - ArchC Platform Manager}}

O \textit{ACPM} é uma evolução e aprimoramento do \textit{ARP}. Sua
proposta é facilitar a obtenção dos componentes, fazendo o
\textit{download} de pacotes a partir de um repositório na web. Sendo
assim, o \textit{ACPM} é uma ferramenta on-line de gerenciamento e
composição de plataformas. O \textit{ACPM} engloba o \textit{ARP} e
expande seu conceito, tornando-se assim uma ferramenta mais versátil e
completa.

Pretende-se neste projeto dar suporte a mais de um repositório, de
forma que os desenvolvedores tenham a flexibilidade de manter versões
diferentes dos componentes (interna e externa) e também utilizar os
componentes de outros usuários do ACPM.

A mudança de nome se deve principalmente pelo fato de já existir um
comando nativo no Linux chamado ARP. Este conflito de nomes já causou
problemas de entendimento e execuções incorretas na versão atual,
sucitando a necessidade de um outro nome.
        
%----------------------------------------------------------------------------------------    
\subsection{\textit{ArchC}}
%----------------------------------------------------------------------------------------
O \textit{ArchC} é uma Linguagem de Descrição de Arquitetura baseada
no \textit{SystemC}, desenvolvido pelo LSC (Laboratório de Sistemas de
Computaçãoo) no IC (Instituto de Computação) na Unicamp (Universidade
Estadual de Campinas), que permite a descrição em alto nível de um
processador, além de oferecer a geração automática de simuladores e
ferramentas de desenvolvimento de software.

Os simuladores gerados por ArchC podem ser utilizados tanto em
sistemas isolados quanto em plataformas. O foco deste projeto é
utilizar os processadores já existentes no ArchC.

%----------------------------------------------------------------------------------------
\subsection{\textit{SystemC}}

\textit{SystemC} é um conjunto de classes \textit{C++} e macros que
fornecem um \textit{kernel} de simulação orientada a eventos em
\textit{C++}. SystemC permite a um projetista simular processos
simultâneos, cada um descrito usando simples sintaxe
\textit{C++}. Processos \textit{SystemC} podem comunicar-se em um
ambiente em tempo real simulado, utilizando sinais de todos os tipos
de dados oferecidos pelo \textit{C++}, algumas entidades adicionais
oferecidos pela biblioteca \textit{SystemC}, além de entidades
definidas pelo utilizador.

Todos os simuladores gerados por ArchC são codificados em SystemC e
todas os componentes das plataformas também o são.
%----------------------------------------------------------------------------------------
\subsection{ESLBench}
O ESLBench é um \textit{benchmark} de plataformas que está sendo
desenvolvido no LSC. As plataformas podem utilizar os processadores
MIPS, PowerPC, SPARC e ARM, além de periféricos comuns entre os
componentes. A versão em desenvolvimento do ESLBench possui softwares
do benchmark SPLASH-2\cite{splash2,parsec-splash}, do ParMiBench\cite{parmibench} e
também do MiBench\cite{mibench}. As plataformas são descritas em
ArchC e SystemC.

O ESLBench será o primeiro grande projeto a fazer uso do
ACPM e o candidato a esta bolsa PIBIC já se encontra trabalhando no
desenvolvimento do ESLBench.

        

%%%%%%%%%%%%%%%%%%%%%%%%%%%%%%%%%
%%%%%                                O Projeto                                %%%%%
%%%%%%%%%%%%%%%%%%%%%%%%%%%%%%%%%    
\section{Projeto}
O projeto é o aprimoramento contínuo do \textit{ACPM}, dando
continuidade ao trabalho desenvolvido no \textit{ARP} e no gerenciador
de plataforma desenvolvido pelo Matheus Nagliati. O \textit{ACPM} será
expandido, desenvolvendo novas funcionalidades e aprimorando as já
existentes. A principal motivação é transformar o \textit{ACPM} num
gerenciador de pacotes similar ao utilizado em distribuições linux, 
embora voltado para as plataformas de sistemas embarcados. Tendo isso
em mente, podemos estabelecer alguns objetivos principais:

\begin{itemize}
\item Aglutinação de componentes num repositório central
\item Suporte a múltiplos repositórios, incluindo uma escala de prioridade dentre os mesmos
\item Suporte a ferramentas, como \textit{cross-compilers} e \textit{debuggers}
\item Criação de uma interface web local para criação de plataformas \textit{on-line} - \textit{WebACPM}
\end{itemize}

\subsection{Aglutinação de componentes}
Esta é a função principal do ACPM, garantir que os pacotes estejam
disponíveis e possam ser baixados por quem necessitar. Desta forma,
através de comandos \textit{get} e \textit{put} será possível buscar
do servidor os componentes necessários para montar uma nova plataforma
localmente permitindo a composição de plataformas distintas.

Contudo, vale ressaltar que a proposta deste projeto não é gerar os
componentes, apenas coletar os já existentes e org permitir o suporte a repositórios
remotos, melhor empacotamento dos componentes, suporte a ferramentas distintas
anizá-los perante as
diretrizes propostas pelo \textit{ACPM}.

\subsection{Múltiplos repositórios}
A \textit{ACPM} trabalha buscando os componentes desejados num
repositório na web. Contudo, a existência de apenas um repositório
implica em uma série de questões e pontos críticos. Por exemplo, caso
o servidor do repositório fique fora do ar, o acesso aos pacotes
ficaria suspenso até que a situação do servidor fosse
normalizada. Outro quesito importante é ter a possibilidade de
organizar um repositório local para uma equipe de desenvolvimento que
também pode querer distribuir alguns de seus componentes num
repositório global.

A solução para essas questões é a utilização de vários
repositórios, compondo uma árvore de prioridades. Desse modo, a
exigência sobre os repositórios seria distribuída, o que tornaria o
\textit{ACPM} mais robusto, já que existiriam mais possibilidades para
a busca e obtenção dos componentes requisitados. Além disso, podem
existir repositórios ``não-oficias'' com componentes criados pelos
próprios usuários e estes podem ser adicionados à listagem de
repositórios locais da \textit{ACPM}.

Em suma, a existência de múltiplos repositórios tornará o
\textit{ACPM} mais próximo da proposta de funcionar como um
gerenciador de pacotes de uma distribuição linux.

\subsection{Suporte a ferramentas}
O objetivo principal do \textit{ACPM} é realizar a gestão dos
componentes para a geração de plataformas de sistemas embarcados. Um
objetivo secundário será a gestão das ferramentas para a simulação
dessas plataformas.

Existe uma série de pré-requisitos a serem atendidos de modo que se
possa rodar uma plataforma descrita em \textit{ArchC}. O \textit{ACPM}
deve verificar a existência desses pré-requisitos e fazer o download e
instalação dos mesmos, caso não estejam sendo atendidos. Ou seja, o
\textit{ACPM} verificaria se existe o \textit{ArchC}, o
\textit{SystemC}, \textit{TLM} e os \textit{cross-compilers}
apropriados e, caso exista alguma insuficiência, ela seria resolvida.

\subsection{\textit{WebACPM}}
O \textit{WebACPM} tem como meta ser uma interface web local, para ser
utiizada através de um navegador de internet, que deve permitir a
geração de plataformas em modo gráfico, facilitando a interligação de
seus componentes e também o melhor uso dos comandos existentes,
evitando as versões de linha de comando. Além
de gerar plataformas, o \textit{WebACPM} deve fazer a gestão dos componentes
locais e também possibilitar a obtenção de novos componentes. Em
última análise, o \textit{WebACPM} será a interface gráfica para o
\textit{ACPM}.

%%%%%%%%%%%%%%%%%%%%%%%%%%%%%%%%%

%%%%%                                Cronograma                            %%%%%
%%%%%%%%%%%%%%%%%%%%%%%%%%%%%%%%%
\section{Cronograma}

A Tabela \ref{tab:cronograma} mostra o cronograma para um ano de
projeto, separado em metas trimestrais. É importante destacar que o
candidato já está trabalhando desde fevereiro de 2012 no projeto e já
domina o código da versão atual da ARP, além de participar ativamente
da definição do ACPM e do ESLBench. O candidato tem participado de
duas reuniões de projeto por semana e também tem interagido com a
doutoranda Liana Duenha, que participa do mesmo laboratório.

\begin{table}[htbp]
\begin{center}
  \begin{tabular}{cc|m{6cm}|m{6cm}|}
    \cline{3-4}
    & & \multicolumn{1}{|c|}{Atividades} & \multicolumn{1}{c|}{Metas}  \\ \cline{1-4}

    \multicolumn{1}{|c||}{\multirow{4}{*}{ \begin{sideways} Um ano de
          Iniciação Científica \end{sideways}}} & 
    \multicolumn{1}{|m{0.3cm}||}{ \begin{sideways} 1º trimestre \ \end{sideways}}
    & % Atividades do 1 trimestre
    \textbullet Conversão do código do ARP para ACPM e inclusão de
    suporte a: múltiplos repositórios, formatos de arquivos zip.

    \textbullet Coleta de componentes para o repositório central.
    & % Metas do 1 trimestre
    \textbullet Lançamento da primeira versão do ACPM.
    \\ \cline{2-4} 

    \multicolumn{1}{|c||}{} &
    \multicolumn{1}{|m{0.3cm}||}{\begin{sideways} 2º trimestre \ \end{sideways}}
    & % Atividades do 2 trimestre
    \textbullet Integração do ESLBench dentro do ACPM e inclusão de
    novas funcionalidades para dar suporte ao benchmark.
    
    & % Metas do 2 trimestre
    \textbullet Distribuição do ESLBench através do ACPM.

    \textbullet Relatório Parcial de Atividades.
    \\ \cline{2-4} 

    \multicolumn{1}{|c||}{} &
    \multicolumn{1}{|m{0.3cm}||}{\begin{sideways} 3º trimestre \ \end{sideways}}
    & % Atividades do 3 trimestre
    \textbullet Suporte a ferramentas dentro do ACPM.
    
    & % Metas do 3 trimestre
    \textbullet Inclusão de ferramentas no repositório do ACPM. \\ \cline{2-4} 
 
   \multicolumn{1}{|c||}{} & \multicolumn{1}{|m{0.3cm}||}{\begin{sideways} 4º trimestre  \ \end{sideways}}
    & % Atividades do 4 trimestre
    \textbullet Desenvolvimento do WebACPM.
    
    & % Metas do 4 trimestre
    \textbullet Distribuição do WebACPM no repositório, como uma ferramenta.
    
    \textbullet Relatório final. \\ \cline{1-4}
  \end{tabular}
  \caption{Cronograma do projeto}
  \label{tab:cronograma}
\end{center}
\end{table}
-------------------------------------------------------------------------------------
\subsection{Publicação}
Além da apresentação no Congresso de Iniciação Científica da Unicamp,
pretende-se ao final deste projeto, enviar um artigo científico sobre
o ACPM para um fórum nacional da área.
%----------------------------------------------------------------------------------------
        
%%%%%%%%%%%%%%%%%%%%%%%%%%%%%%%%%
%%%%%                        Sobre o candidato                        %%%%%
%%%%%%%%%%%%%%%%%%%%%%%%%%%%%%%%%
\section{Sobre o candidato}
Meu chamo Matheus Ferreira Tavares Boy, ingressei em 2010 na Unicamp
no curso de Engenharia de Computação. Tive 3 reprovações até agora, no
primeiro e no terceiro semestre. Reprovei em Física I, Física II e
Geometria Analítica. Contudo, são disciplinas do primeiro ano de
curso, e atribuo as reprovações a questões na adaptação à vida
universitária. Além disso, ressalto que vou bem na parte específica de
computação do curso, sem ter tido reprovações nestas disciplinas, e
obtendo notas acima da média da turma na maioria das disciplinas de
computação.

%%%%%%%%%%%%%%%%%%%%%%%%%%%%%%%%%
%%%%%                                Bibliografia                            %%%%%
%%%%%%%%%%%%%%%%%%%%%%%%%%%%%%%%%

\bibliographystyle{plain}
\bibliography{Projeto}

\end{document}
