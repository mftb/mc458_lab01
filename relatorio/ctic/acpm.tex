\documentclass[12pt]{article}

\usepackage{sbc-template}

\usepackage{graphicx,url}

%\usepackage[brazil]{babel}   
\usepackage[utf8]{inputenc}  

     
\sloppy

\title{ArchC Platform Manager - Um gerenciador de pacotes para plataformas de
sistemas embarcados}

\author{Matheus Ferreira Boy\inst{1}, Rodolfo Azevedo\inst{1} }


\address{Instituto de Comutação -- Universidade Estadual de Campinas -- UNICAMP\\
  Av. Albert Einstein, 1251 -- Campinas -- SP -- Brasil -- 13083-852
  \email{matheus.boy@students.ic.unicamp.br, rodolfo@ic.unicamp.br}
}

\begin{document} 

\maketitle

\begin{abstract}
\end{abstract}
     
\begin{resumo} 
Este artigo apresenta a ferramenta ArchC Platform Manager (ACPM), criada para
auxiliar o gerenciamento de projeto e desenvolvimento de sistemas embarcados. O
ACPM permite organizar os componentes locais, além de facilitar a obtenção e distribuição de
componentes através de repositórios web. O ACPM é a ferramenta sucessora
do ArchC Reference Platform (ARP) e as melhorias implementadas no ACPM são o
foco deste artigo.

\end{resumo}


\section{General Information}




\section{Introdução}

O desenvolvimento e projeto de sistemas embarcados requer a criação de múltiplas
versões do mesmo sistema, podendo variar desde alguns detalhes até grandes
componentes. Gerenciar essas múltiplas versões não é algo trivial e é parte
crucial no desenvolvimento de novos sistemas embarcados. Pensando nisso, foi
desenvolvido o ArchC Platform Manager (ACPM), sucessor do ArchC Reference
Platform (ARP), ambas ferramentas desenvolvidas no Laboratório de Sistemas de
Computação (LSC) do Instituto de Computação (IC) da Universidade Estadual de
Campinas (Unicamp).

O ARP era um gerenciador de plataformas que consistia em uma estrutura de
diretórios, um Makefile padrão do ARP e um software utilitário que permitia
empacotar e desempacotar componentes.

O ACPM veio como um sucessor aprimorado do ARP e sua expansão e aprimoramento 
são o foco da atividade de Iniciação Científica. Essencialmente, o ACPM é uma
estrutura de diretórios associada a um software gerenciador de pacotes de
plataformas de sistemas embarcados. A principal função do ACPM é organizar os
componentes numa estrutura sistemática e possibilitar a obtenção de novos
componentes para geração de plataformas. A obtenção de componentes se dá através
dos mesmos já empacotados, através de \textit{download} de repositório na web. Após o
\textit{download}, o software desempacota o componente, estando este pronto para uso.
Essa é a principal diferença do ACPM para o ARP. O ACPM consegue obter novos
componentes (pacotes) on-line, enquanto o ARP não possui essa funcionalidade.
Novass funcionalidades foram inseridas no ACPM e outras que já existiam no ARP
foram aprimoradas. Maiores detalhes serão dados mais adiante neste artigo.

Um trabalho relacionado ao ACPM é o ESLBench, também desenvolvido no LSC.
O ESLBench é um \textit{benchmark} para medir o desempenho de plataformas descritas em
ArchC e SystemC, e utilizou diversos recursos do ACPM, como a estrutura de
diretórios e os esquemas de Makefile próprios do ARP/ACPM. 

O artigo é organizado da seguinte forma. A seção 2 apresenta a ferramenta ACPM, a
seção 3 apresenta as melhorias feitas na ferramenta durante a atividade de
Iniciação Científica, a seção 4 apresenta os resultados e a seção 5 conclui.


\section{Visão Geral da Ferramenta}

O ACPM organiza os componentes das plataformas em uma estrutura com 6
diretórios, e cada componente possui seu respectivo subdiretório. A estrutura de
diretórios organiza os componentes nas seguintes categorias:
    - processors: Diretório que contém os modelos de processadores.
    - is: Diretório que contém os elementos de interconexão disponíveis para a
        construção de plataformas, como barramentos, roteadores, NoCs, etc.
    - ip: Diretório que contém os IPs (Intellectual Properties) disponíveis. 
        Cada IP colocado neste diretório se torna disponível para utilização em
        plataformas.
    - sw: Diretório que contém os softwares que serão rodados nas plataformas.
        Um mesmo software pode ser executado em mais de uma plataforma, embora
        isso não seja muito comum.
    - wrappers: Diretório que contém os conversores de protocolos que podem ser
        utilizados para conectar componentes distintos das plataformas.
    - platforms: Diretório que contém as plataformas desenvolvidas. Cada versão
        diferente da plataforma deve possuir um diretório diferente, facilitando
        o gerenciamento das mesmas.

Essa estrutura é herança direta do ARP. Vale ressaltar que a classificação e
separação dos componentes é o que permite a reutilização inteligente dos mesmos.
Por exemplo, se existem duas plataformas, uma multicore e outra singlecore, cada
uma terá seu respectivo diretório dentro de platforms, e cada plataforma
instanciará o número apropriado de processadores no seu respectivo código fonte,
assim como os demais componentes (IPs, barramentos, wrappers).

Ao se realizar o download da ferramenta é necessário iniciá-la através de um
comando específico. Feito isso, o ACPM cria a estrutura básica de diretórios,
cria dois arquivos de configuração, um para os endereços de repositórios web e
outro para o repositório local, e baixa o Makefile. Assim, a ferramenta já está
pronta para uso. O software, em linhas gerais, permite:
    - obtenção de novos componentes;
    - empacotamento de componentes desenvolvidos;
    - listagem de componentes disponíveis localmente;
    - listagem de componentes disponíveis no repositório web;
    - geração de templates;
    - suporte a múltiplos repositórios web;
    - submissão de pacotes para um repositório.

Essas são as principais funcionalidades atualmente presentes no ACPM, contudo,
isso não quer dizer que elas sempre fizeram parte da ferramenta, ou que sempre
funcionaram da mesma forma. Todas sofreram modificações decorrentes do trabalho
desenvolvido na atividade de Iniciação Científica, e a ferramenta como um todo
foi amplamente expandida e melhorada.

\section{Melhorias e Expansões da Ferramenta}

Inicialmente, o trabalho de melhoria do ACPM se deu através do estudo e análise
do código original do gerenciador do ARP. Após devida compreensão e detecção de
pontos de interesse para melhoria ou modificação no código, iniciaram-se os
trabalhos de modificação no gerenciador.

Um ponto interessante notado desde o princípio foi que o gerenciador da ARP foi
originalmente implementado em Python, e escolheu-se manter essa como ferramenta
de implementação do ACPM, devido à praticidade dessa linguagem para
implementações rápidas. Outra característica boa do Python é a vasta biblioteca
padrão da linguagem, facilitando tarefas como geração de arquivos zip, chamadas
para o interpretador de comandos do sistema operacional e acesso à web.

Após extenso estudo do código do gerenciador, foram detectados muitos pontos
passíveis de melhoria. Existia um tratamento de exceções muito superficial,
tornando o código pouco tolerante a falhas. Na maior parte das vezes, quando era
lançada uma exceção, eram emitidas verbosas mensagens de erro do interpretador
do Python, que não davam pistas sobre o real problema que gerou a exceção. Por
exemplo, apesar de ser um gerenciador de pacotes que buscava recursos na web,
nenhum método que buscava dados on-line estava dentro de um bloco de tratamento
de exceções. Além disso, nenhuma outra operação de I/O estava com suas exceções
devidamente tratadas.

Depois de detectados esses pontos de interesse, foi iniciado um minucioso e
extensivo trabalho de blindagem do código. Todos os pontos passíveis de exceção
foram inseridos em blocos try/except, inclusive diferenciando-se alguns tipos
específicos de exceções de modo a receberem tratamento apropriado ao invés de um
tratamento genérico e inespecífico. O processo de blindagem do código foi
gradual e iterativo. Inicialmente, o tratamento de exceções se resumia a emitir
uma mensagem mais clara ao usuário do que uma mera exceção do interpretador do
Python. Posteriormente, o tratamento de exceções foi evoluindo de modo que o
gerenciador continuava sua execução mesmo com o surgimento de uma exceção.
Depois desse processo, a versão original do software gerenciador foi declarada
como blindada e tolerante a falhas, além de estar apta a receber novas
funcionalidades e outras melhorias no código.

Além das correções quanto ao tratamento de exceções, foram corrigidos alguns
aspectos do código original do gerenciador. Algumas funções redefiniam funções
builtin do Python, como list(), mas sem expandir o funcionamento dessa função.
Desse modo, os nomes das funções foram trocados de modo que nada da linguagem
fosse redefinido.

A primeira funcionalidade a ser incluída foi o suporte a múltiplos repositórios.
Isso foi implementado de modo que sempre que o gerenciador for rodado ele
seleciona o primeiro repositório válido no arquivo de lista de repositórios web.
Esse arquivo é inicialmente gerado quando se inicia o ACPM após o download do
mesmo, e vem com apenas o endereço do repositório padrão. Esse arquivo é
facilmente editável pelo usuário, podendo acrescentar outros repositórios
conforme lhe aprouver. O arquivo funciona como uma fila de prioridades.
Portanto, o repositório que estiver na primeira linha possui a maior prioridade,
na segunda linha, a segunda maior prioridade e assim sucessivamente. A seleção
automática de repositório tem um funcionamento relativamente simples.
Essencialmente, o gerenciador tenta acessar sequencialmente cada repositório
listado no arquivo, e seleciona o primeiro que retornar uma resposta HTTP 200
(OK). Se nenhum repositório retornar uma HTTP 200, o repositório padrão,
hardcoded, é selecionado.

Depois disso, foi proposta a mudança do formato de empacotamento dos componentes
de tar para zip. Isso se deu devido ao fato do gerenciador do ARP originalmente
gerar os pacotes chamando tar(1) num subprocesso aberto pelo interpretador de
comandos do sistema operacional. Essa decisão, contudo, restringe o uso do
gerenciador a um ambiente POSIX. Como o Python suporta nativamente arquivos zip
através de uma biblioteca padrão, a modificação dessa funcionalidade foi
relativamente simples. Foi criada uma classe auxiliar para facilitar o manuseio
de tais arquivos, e a inicialização do gerenciador após o download do mesmo
precisou ser alterada. Isso se deu devido ao fato de originalmente ser realizado
o download de um arquivo tar com a estrutura básica de diretórios durante a
inicialização do ACPM, ao passo que o formato zip não permite explicitamente a
compactação de uma árvore de diretórios. O que se faz para comprimir uma
estrutura de diretórios usando zip é renomear os arquivos de modo a concatenar o
path de diretórios do arquivo com o próprio nome do arquivo. Contudo, isso não
permite adicionar diretórios vazios a um arquivo zip, de modo que a estrutura
básica do ARP precisaria ser gerada de outra forma na inicialização do
gerenciador. Isso foi resolvido de modo que os diretórios são criados
dinamicamente na função de inicialização do gerenciador, através da biblioteca
padrão do Python. Resolvida essa questão, foi modificada a funcionalidade de
empacotamento de componentes de modo a comprimir os arquivos para zip ao invés
de tar.

Outra funcionalidade proposta foi a implementação de um parser de YAML no
próprio código do gerenciador. O ACPM usa YAML na listagem de componentes
disponíveis nos repositórios web. YAML é um formato de serialização de
codificação de dados facilmente legível e editável por seres humanos.
Originalmente, o parsing era feito utilizando-se uma biblioteca desenvolvida
pelo próprio projeto YAML, contudo, não padrão do Python. Sendo assim, para o
gerenciador funcionar de forma correta, era necessária a instalação dessa
biblioteca, algo não-trivial num ambiente não-POSIX. A implementação do parser
de YAML foi uma tarefa não muito trivial, visto que seu principal desafio não
residia no conhecimento da biblioteca padrão do Python, mas sim no capacidade
de projeto e desenvolvimento de algoritmos por parte do aluno. Sendo assim, foi
um dos problemas mais interessantes resolvidos no ACPM. Essencialmente, lê-se o
arquivo YAML e monta-se uma hashtable (dicionário) a partir do arquivo. O parser
foi implementado com sucesso e é parte essencial da atual versão do ACPM.
Laboratório de Projetos de Sistemas Computacionais

Implementou-se uma forma inicial de um "put", ou seja, submissão de pacotes para
um repositório. A forma inicial não permite, explicitamente, a submissão de
componentes para um repositório web, embora o usuário possa fazer com que essa
funcionalidade se comporte assim. Essencialmente, o put foi implementado com uma
chamada para a função empacotadora passando como argumento um componente
específico, e a subsequente cópia do componente empacotado para um determinado
diretório. O diretório-alvo do put pode ser alterado através de um arquivo de
configuração, gerado automaticamente pelo ACPM, mas de fácil edição, de modo que
o usuário possa especificar um local mais apropriado para submissão de seus
componentes. Desse modo, o usuário pode alterar o arquivo de configuração para
apontar para um diretório que seja de fato um repositório web.

Foi feita uma alteração na funcionalidade de empacotamento de componentes.
Incluiu-se a opção de se empacotar todos os componentes, separadamente, com
apenas um único comando. Originalmente, podia-se empacotar apenas um componente
por vez, ou então todos os componentes em um único pacote. Agora, pode-se
empacotar todos os componentes em seus respectivos pacotes individuas de uma vez
só. Isso foi implementado utilizando-se chamadas recursivas na função
empacotadora à medida que se caminha na estrutura padrão do ACPM (platforms,
processors, ip, is, sw e wrappers), passando como argumento o tipo do componente
e o próprio componente encontrado. Contudo, manteve-se a possibilidade de gerar
um único arquivo contendo todos os componentes empacotados juntos.

A última modificação feita foi no Makefile do ARP. Originalmente, o Makefile era
um único arquivo. No ACPM, o Makefile divide-se em 3 arquivos: Makefile,
Makefile.rules e Makefile.conf. No Makefile está a plataforma que se deseja
executar. No Makefile.rules estão as regras de compilação e execução comuns a
todos os componentes e plataformas gerenciados pelo ACPM. No Makefile.conf estão
os paths para o SystemC, ArchC e TLM, além da variável de ambiente que define o
sistema operacional do usuário. Essa ideia de separação do Makefile do ARP em 3
foi expandida no ESLBench, visto que o ESLBench gera dinamicamente o arquivo
Makefile para cada uma de suas plataformas a serem testadas. Essa separação do
Makefile facilita a edição do usuário, visto que ele só precisa alterar um único
arquivo simples para mudar a plataforma que ele deseja executar ou compilar no
momento. Isso também facilita a edição dos paths para as ferramentas de
compilação (SystemC, ArchC, TLM), visto que isso normalmente varia de máquina
para máquina. O único arquivo que o usuário não precisa editar é o
Makefile.rules, visto que este arquivo contém as regras de compilação e execução
das plataformas segundo o que foi definido no ARP. Esses 3 arquivos de Makefile
funcionam de modo que o Makefile principal dá um include nos outros 2 arquivos,
atuando perante o make(1) como se fosse um único grande arquivo de Makefile.
Essencialmente, essa foi uma decisão de projeto visando melhorar a experiência
do usuário com o ACPM.


\section{Resultados}

O principal resultado, até agora, é a primeira versão do ACPM, que se encontra
disponível para download no site do ArchC. Um paper relacionado ao ESLBench foi
submetido ao periódico do IEEE "Embedded Systems Letters".

Além disso, a estrutura básica de diretórios do ACPM é amplamente utilizada no
LSC no desenvolvimento e pesquisa de sistemas embarcados. A estrutura também é
utilizada na disciplina de Laboratório de Projetos de Sistemas Computacionais,
oferecida no curso de Engenharia de Computação da Unicamp.
As funcionalidades de obtenção de componentes do ACPM começarão a ser mais
utilizadas assim que forem coletados mais compenentes para serem
disponibilizados no repositório padrão da ferramenta.


\section{Conclusões}

Gerenciar as diversas versões de um sistema embarcado pode levar inconsistências
entre os componentes, gerando diversos erros. Além disso, a distribuição
descentralizada de componentes também colabora para essas mesmas incosistências.
A ferramenta ACPM surge para fornecer um ambiente de trabalho organizado e
estruturado, além de permitir a fácil obtenção de componentes de uma fonte
confiável através da distribuição por repositórios web.

A ferramenta está em constante desenvolvimento e expansão, e serão inclusas
funcionalidades de distribuição de ferramentas como o SystemC e ArchC através
do ACPM, além de uma interface gráfica para maior facilidade de uso da
ferramenta. 

O ACPM se encontra para download no site do ArchC, assim como alguns componentes
e plataformas de exemplo. O repositório padrão atual está em expansão, e em
breve contará com diversos componentes e plataformas, prontos para uso
educacional ou científico.


\section{Agradecimentos}

Este trabalho foi financiado com recursos do CNPq.


\bibliographystyle{sbc}
\bibliography{sbc-template}

\end{document}
